\documentclass[14pt]{beamer}
\usetheme{metropolis}
\usepackage{marvosym}
\usepackage{xcolor}
\usepackage{graphicx,txfonts}
\newcommand{\heart}{\ensuremath\varheartsuit}
\newcommand{\butt}{\rotatebox[origin=c]{180}{\heart}}
\usepackage[slantfont,boldfont]{xeCJK}

\title{教学相关材料编写指引}
\date{2020年11月3日}
\author{吴甜甜}
\institute{翻译教研室@外国语学院}
\begin{document}
\maketitle
\begin{frame}{目录}
  \setbeamertemplate{section in toc}[sections numbered]
  \tableofcontents[hideallsubsections]
\end{frame}
\section{教学材料的内容和版本说明}
\begin{frame}
  \frametitle{教学材料包括}
  \begin{enumerate}
  \item 课程教学大纲
  \item 课程考核大纲
  \item 实验教学大纲
  \item 课程实习大纲
  \item 实习实训大纲
  \item 授课计划
  \item 实验教学任务书
  \item 教案
  \end{enumerate}
\end{frame}

\section{教学材料编写的共性要求}
\subsection{关于大纲中课程基本情况的说明}
\begin{frame}
  \frametitle{大纲中课程基本情况}
  \includegraphics[width=10cm]{1.png}
\end{frame}
\begin{frame}
  \frametitle{商务英语情况}
  商务英语人培共有三个版本,分别是15级,16级,19级。
  其中:
  \begin{enumerate}
  \item 15级为单独一个版本
  \item 16-18为一个版本
  \item 19级以后为一个版本
  \end{enumerate}
  如果课程名称一致,但三个版本的部分课程代码有所变动的话,理论学时、实验学时,总学时通常有所调整,需要注意与人培方案保持一致。
\end{frame}

\begin{frame}
  \frametitle{15级情况}
  15级大纲由第一次代课老师撰写和修改。
\end{frame}

\begin{frame}
  \frametitle{16-18级大纲可继续沿用15级大纲的情况}
  \begin{alertblock}{沿用}
    \begin{enumerate}
    \item 如果和15级大纲的课程代码一致,未换教材,不需要重写。
    \item 如果代码一致,却换了新教材,新教材与旧教材吻合度超过60\%,也无需重写。
    \end{enumerate}
  \end{alertblock}
\end{frame}

\begin{frame}
  \frametitle{16-18级大纲不可继续沿用15级大纲的情况}
  \begin{alertblock}{不可沿用}
    \begin{enumerate}
    \item 如果和15级大纲的课程代码不一致,需要重写大纲,并注意学分、学时、理论学时和实践学时的调整,并在大纲的学时设置里做相应的变动。
    \item 如果和15级大纲的课程代码一致,但是换了教材,并且新教材和就教材的吻合率较低,不超过60\%,就需要重写大纲。
    \end{enumerate}
  \end{alertblock}
\end{frame}

\begin{frame}
  \frametitle{翻译专业情况}
  \begin{alertblock}{翻译专业人才培养计划}
    翻译专业的人培计划有两个版本,18级和19级,18级大纲按照目前实际的代课情况进行
    编写,\emph{19级的大纲已完成},请授课教师根据人培方案进行修订。
  \end{alertblock}
\end{frame}

\begin{frame}
  \frametitle{大学英语}
  \begin{alertblock}{大学英语情况}
    大学英语需要编写两个版本,分别是15-18级,19级。
  \end{alertblock}
\end{frame}

\subsection{关于教材变动的说明}
\begin{frame}
  \frametitle{关于教材变动的说明}
  16-18级应该为同一个大纲系列,但有些课程换了教材,可以挑选不同教材的同质性内容,进行编写。
\end{frame}

\subsection{关于教学材料一致性的说明}
\subsubsection{各项大纲与人才培养方案的一致性}
\begin{frame}
  \frametitle{各项大纲与人才培养方案的一致性}
  \begin{alertblock}{基本信息}
    各项大纲中的课程基本信息(课程代码、课程性质、总学时、理论学时、实验学时等)应与人才培养方案中的表2,表3保持一致。
  \end{alertblock}
  \begin{exampleblock}{其他信息}
    各项大纲中的教学目的、课程说明、课程教学目标、考核目标等内容也应围绕人才培养方案,与人才培养方案内容保持一致,避免出现复合型人才的字样,突出应用型人才的培养。
  \end{exampleblock}
\end{frame}

\subsubsection{教学材料间的一致性}
\begin{frame}
  \frametitle{教学材料间的一致性}
  \begin{enumerate}
  \item 各种资料学时分配合理
  \item 内容的一致性
  \item 考核方式的一致性
  \end{enumerate}
\end{frame}

\begin{frame}
  \frametitle{学时分配}
  教学大纲中的理论学时、实验学时的设置,要合理,授课计划(教学进度表)、教案中的学
  时设置,实验教学任务书里的实验学时分配都要与教学大纲保持一致。
  请看下面的图片:
\end{frame}

\begin{frame}
  \frametitle{教学大纲}
  \includegraphics[width=10cm]{2.png}x
\end{frame}

\begin{frame}
  \frametitle{实验教学大纲}
  \includegraphics[width=10cm]{3.png}
\end{frame}

\begin{frame}
  \frametitle{教学进度表}
  \includegraphics[width=10cm]{4.png}
\end{frame}

\begin{frame}
  \frametitle{实验教学任务书}
  \includegraphics[width=10cm]{5.png}
\end{frame}

\begin{frame}
  \frametitle{内容一致}
  \begin{alertblock}{内容}
    考核大纲的内容应与教学大纲的内容相对应,需要了解、理解、掌握等内容基本一致。
  \end{alertblock}
\end{frame}

\begin{frame}
  \frametitle{考核方式一致}
  \begin{alertblock}{考核方式}
    教学大纲、考核大纲中的考核方式、所占比例,应当与实际保持一致。
  \end{alertblock}
\end{frame}

\subsection{关于个别材料的要求说明}
\begin{frame}[allowframebreaks]
  \frametitle{15-18级考核大纲中试题结构和题型比例}
  试题比例可采取区间的形式,但是应与实际的考试题相符合。
  \begin{exampleblock}{试题比例}
    如第一章15\%-20\%,第二章20\%-30\%......
  \end{exampleblock}
  \framebreak
  \begin{exampleblock}{}
    考试题型最好无变动,如果有变动的话,可以说明在几种题型中任选:
  \end{exampleblock}
  \begin{alertblock}{考题变动情况}
    如该试题在短语翻译20\%,连线题20\%,判断正误题20\%,简答题20\%,案例分析20\%,单选题20\%等题型中任
  \end{alertblock}
\end{frame}
\begin{frame}
  \frametitle{实验大纲与实验教学任务书}
  \begin{enumerate}
  \item 按照评估要求,实验类型分别为演示、验证、综合、设计、创新,其中演示与验证要尽量少,综合、设计和创新比例尽量要高。
  \item 参照国家质量标准要求,实验人数尽量在10人以内,最好是6-8人。
  \end{enumerate}
\end{frame}
\begin{frame}[allowframebreaks]
  \frametitle{关于排版和格式的要求}
  \begin{enumerate}
  \item 首先要注意字符的字体、缩进、空格等保持一致。
  \item 为保持一致性,大纲编写中,模板中的序号、标点、黑体、排版尽量不要改动。
  \item 页面设置:15-18版的大纲模板中对页面设置未做要求(为保持一致,建议采用19级模板的页面设置方式)。
  \end{enumerate}
  \framebreak
  \begin{exampleblock}{19级大纲}
    19级的大纲模板中对页面设置的要求为上下边距2.5cm,左边距3cm,右边距2.5cm,左订线0.5cm ,全文1.5倍行距,段前段后0行,正文字体全部宋体。
  \end{exampleblock}
\end{frame}
\section{15-18级和19级材料编写的区别}
\begin{frame}
  \frametitle{15-18级大纲}
  采用同一个版本,要求按课本章节编写,但是章节的筛选,整合由任课教师根据人才培养方案的培养目标和培养要求进行灵活掌握。
\end{frame}

\begin{frame}
  \frametitle{19级大纲}
  采用19级大纲新版本,不按照课本章节编写,按照培养目标进行编写,突出OBE的模式,
  为了达成培养目标,设置哪些章节,需要补充什么样的材料,都由教师自行决定。做到以
  课本为依托,但是不拘泥于课本。
  举个例子:
\end{frame}

\begin{frame}[allowframebreaks]
  \frametitle{举个例子}
  以英美文学为例,如英美文学的要培养学生的阅读能力,语言表达能力,文学作品的批评能
  力,增强对西方社会文化知识的理解,感受文学艺术的魅力,提高文学修养和思辨能力等。
  按照这个培养目标,对知识进行整合,将课本知识整合为以下几个章节:
  \framebreak
  \begin{enumerate}
  \item 英语小说(阅读能力的培养、文学批评理论的理解)
  \item 英语戏剧 (语言表达能力,西方社会文化知识)
  \item 英语诗歌(语言的韵律美)
  \item 英语散文(阅读能力,文学修养和思辨能力)
  \end{enumerate}
\end{frame}

\begin{frame}
  \frametitle{关于人才培养方案版本问题}
  商英15-18级和翻译18级人才培养方案已经定稿,不再调整。两个专业19级人才培养方案仍在调整和修订中,旧的版本中两个专业中一些同样的基础课程编码不一致,要求大纲为两个版本。调整后的人才培养方案中,同样的课程编码一致,课程设置也一致,大纲也应为同一个版本。
\end{frame}

\begin{frame}[standout]
  \vspace{0.3in}
  谢谢大家\hspace{0.07in} {\color{red} \heart}
  \\

  \vspace{1in}
  \tiny{Created with \LaTeX{} }
\end{frame}

\end{document}